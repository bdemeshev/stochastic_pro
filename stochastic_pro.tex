% arara: xelatex: {shell: yes}
% arara: biber
% arara: xelatex: {shell: yes}
% arara: xelatex: {shell: yes}



%!TeX cleanPatterns = $OUTDIR/$JOB!($OUTEXT|.synctex.gz|.tex|.pdf), /$OUTDIR/_minted-$JOB/
\documentclass[12pt, a4paper]{article}

% utf8 is the preferred encoding

 % this magick is to solve problem that appeared after update of texlive 2018 to texlive 2020
 % https://tex.stackexchange.com/questions/511341/the-error-occurred-after-the-last-update
\makeatletter
\def\nobreak{\penalty\@M}
\makeatother


\usepackage{fontspec} % что-то про шрифты? % нужно ли загружать?

\usepackage{polyglossia} % русификация xelatex
\usepackage{csquotes}


\setmainlanguage{english}
\setotherlanguage{russian}

% download "Linux Libertine" fonts:
% http://www.linuxlibertine.org/index.php?id=91&L=1
\setmainfont{Linux Libertine O} % or Helvetica, Arial, Cambria
% why do we need \newfontfamily:
% http://tex.stackexchange.com/questions/91507/
\newfontfamily{\cyrillicfonttt}{Linux Libertine O}

\newfontfamily\arabicfont[Script=Arabic]{Scheherazade New}


\usepackage{etoolbox} % provides \AtEndPreamble
% etoolbox causes wrong behavior of tocbasic
\AtEndPreamble{ % ради арабского написания Абу ибн-Сина
  \usepackage{arabxetex} 
 \let\textarabic\relax 
 \let\Arabic\relax 
\setotherlanguages{arabic, english}
}
% комбо из:
% https://tex.stackexchange.com/questions/501897
% https://tex.stackexchange.com/questions/392175/

\usepackage{imakeidx} 
\indexsetup{level=\section}
\makeindex[title=Hashtags]

\usepackage{etex} % расширение классического tex
% в частности позволяет подгружать гораздо больше пакетов, чем мы и займёмся далее

\usepackage{verbatim} % для многострочных комментариев
\usepackage{makeidx} % для создания предметных указателей

\usepackage{setspace}
\usepackage{amsmath, amsfonts, amssymb, amsthm}
\usepackage{mathrsfs} % sudo yum install texlive-rsfs
\usepackage{dsfont} % sudo yum install texlive-doublestroke
\usepackage{array, multicol, multirow, bigstrut} % sudo yum install texlive-multirow
\usepackage{indentfirst} % установка отступа в первом абзаце главы


\usepackage{bm}
\usepackage{bbm} % шрифт с двойными буквами
%\usepackage[perpage]{footmisc}

\usepackage{dcolumn} % центрирование по разделителю для apsrtable

% создание гиперссылок в pdf
\usepackage[unicode, colorlinks=true, urlcolor=blue, hyperindex, breaklinks]{hyperref}


\usepackage{microtype} % свешиваем пунктуацию
% теперь знаки пунктуации могут вылезать за правую границу текста, при этом текст выглядит ровнее


\usepackage{textcomp}  % Чтобы в формулах можно было русские буквы писать через \text{}

% размер листа бумаги
%\usepackage[paperwidth=145mm,paperheight=215mm,
%height=182mm,width=113mm,top=20mm,includefoot]%{geometry}
\usepackage[paper=a4paper, top=15mm, bottom=13.5mm, left=16.5mm, right=13.5mm, includefoot]{geometry}

\usepackage{xcolor}


% \usepackage{float, longtable}
% \usepackage{soulutf8}

\usepackage{enumitem} % дополнительные плюшки для списков
%  например \begin{enumerate}[resume] позволяет продолжить нумерацию в новом списке

\usepackage{mathtools}
\usepackage{cancel, xspace} % sudo yum install texlive-cancel


\usepackage{numprint} % sudo yum install texlive-numprint
\npthousandsep{,}\npthousandthpartsep{}\npdecimalsign{.}


% \usepackage{subfigure} % для создания нескольких рисунков внутри одного

\usepackage{tikz, pgfplots} % язык для рисования графики из latex'a
\pgfplotsset{compat=1.16}
\usetikzlibrary{trees} % tikz-прибамбас для рисовки деревьев
\usepackage{tikz-qtree} % альтернативный tikz-прибамбас для рисовки деревьев
\usetikzlibrary{arrows} % tikz-прибамбас для рисовки стрелочек подлиннее

\usepackage{todonotes} % для вставки в документ заметок о том, что осталось сделать
% \todo{Здесь надо коэффициенты исправить}
% \missingfigure{Здесь будет Последний день Помпеи}
% \listoftodos --- печатает все поставленные \todo'шки



\usepackage{booktabs} %  красивые таблицы
% заповеди из докупентации:
% 1. Не используйте вертикальные линни
% 2. Не используйте двойные линии
% 3. Единицы измерения - в шапку таблицы
% 4. Не сокращайте .1 вместо 0.1
% 5. Повторяющееся значение повторяйте, а не говорите "то же"

\usepackage{physics}
% \usepackage{minted} % moved to listings to simplify development
\usepackage{listings}
\lstset{%
basicstyle=\fontfamily{lmtt}\bfseries,
keywordstyle=\fontfamily{lmtt}\bfseries
}
% \usepackage{julia-mono-listings}
% TODO: установить?
\usepackage{answers}




\usepackage[bibencoding=auto, backend=biber, sorting=none, style=alphabetic]{biblatex}

\addbibresource{stochastic_pro.bib}

\setcounter{tocdepth}{1} % в оглавление оставляем уровень 1

\usepackage[titles]{tocloft} % альтернатива tocbasic для настройки toc
% если нужен subfigure, то у tocloft можно добавить опцию subfigure
\renewcommand{\cftbeforesecskip}{0.7pt} % поправка интервала между строками для section в toc
\renewcommand{\cftsecdotsep}{\cftdotsep} % добавляем точечки

\AddEnumerateCounter{\asbuk}{\russian@alph}{щ} % для списков с русскими буквами
% \setlist[enumerate, 1]{label=\asbuk*),ref=\asbuk*} % цифра рядом с enumerate = уровень нумерации
\setlist[enumerate, 1]{label=\alph*),ref=\alph*} % цифра рядом с enumerate = уровень нумерации


%%%%%%%%%%%%%%%%%%%%%%%  ПАРАМЕТРЫ  %%%%%%%%%%%%%%%%%%%%%%%%%%%%%%%%%%
\setstretch{1}                          % Межстрочный интервал
\flushbottom                            % Эта команда заставляет LaTeX чуть растягивать строки, чтобы получить идеально прямоугольную страницу
\righthyphenmin=2                       % Разрешение переноса двух и более символов
%\pagestyle{plain}                       % Нумерация страниц снизу по центру.
\widowpenalty=300                     % Небольшое наказание за вдовствующую строку (одна строка абзаца на этой странице, остальное --- на следующей)
\clubpenalty=3000                     % Приличное наказание за сиротствующую строку (омерзительно висящая одинокая строка в начале страницы)
\setlength{\parindent}{1.5em}           % Красная строка.
%\captiondelim{. }
\setlength{\topsep}{0pt}
\emergencystretch=2em

% делаем короче интервал в списках
\setlength{\itemsep}{0pt}
\setlength{\parskip}{0pt}
\setlength{\parsep}{0pt}



\DeclareMathOperator{\card}{card}
\DeclareMathOperator{\sign}{sign}
\DeclareMathOperator{\sgn}{sign}

\DeclareMathOperator*{\argmin}{arg\,min}
\DeclareMathOperator*{\amn}{arg\,min}
\DeclareMathOperator*{\amx}{arg\,max}


\DeclareMathOperator{\Corr}{Corr}
\DeclareMathOperator{\sCorr}{sCorr}
\DeclareMathOperator{\sCov}{sCov}
\DeclareMathOperator{\sVar}{sVar}

\DeclareMathOperator{\Cov}{Cov}
\DeclareMathOperator{\Var}{Var}
\DeclareMathOperator{\cov}{Cov}
\DeclareMathOperator{\Bin}{Bin}
\DeclareMathOperator*{\plim}{plim}
\DeclareMathOperator{\MSE}{MSE}
\DeclareMathOperator{\softmax}{softmax}
\DeclareMathOperator{\Med}{Med}


\renewcommand{\P}{\mathbb{P}}
\newcommand{\E}{\mathbb{E}}

\newcommand{\e}{\varepsilon}

\newcommand{\cN}{\mathcal{N}}
\newcommand{\dNorm}{\mathcal{N}}
\newcommand{\dN}{\mathcal{N}}
\newcommand{\dLN}{\mathcal{LN}}

\newcommand{\dBern}{\mathrm{Bern}}
\newcommand{\dPois}{\mathrm{Pois}}
\newcommand{\dBin}{\mathrm{Bin}}
\newcommand{\dMult}{\mathrm{Mult}}
\newcommand{\dGeom}{\mathrm{Geom}}
\newcommand{\dNHGeom}{\mathrm{NHGeom}}
\newcommand{\dHGeom}{\mathrm{HGeom}}
\newcommand{\dDUnif}{\mathrm{DUnif}}
\newcommand{\dFS}{\mathrm{FS}}
\newcommand{\dNBin}{\mathrm{NBin}}

\newcommand{\dTri}{\mathrm{Triangle}}
\newcommand{\dUnif}{\mathrm{Unif}}
\newcommand{\dCauchy}{\mathrm{Cauchy}}
\newcommand{\dExpo}{\mathrm{Expo}}
\newcommand{\dBeta}{\mathrm{Beta}}
\newcommand{\dGamma}{\mathrm{Gamma}}
\newcommand{\dWei}{\mathrm{Wei}}
\newcommand{\dLogistic}{\mathrm{Logistic}}
\newcommand{\dRayleigh}{\mathrm{Rayleigh}}
\newcommand{\dPareto}{\mathrm{Pareto}}


% вместо горизонтальной делаем косую черточку в нестрогих неравенствах
\renewcommand{\le}{\leqslant}
\renewcommand{\ge}{\geqslant}
\renewcommand{\leq}{\leqslant}
\renewcommand{\geq}{\geqslant}


\newcommand{\wv}{\textrm{word2vec}}
\newcommand \hVar{\widehat{\Var}}
\newcommand \hCorr{\widehat{\Corr}}
\newcommand \hCov{\widehat{\Cov}}


\newcommand{\RR}{\mathbb{R}}

\newcommand{\Lin}{\mathcal{L}in}
\newcommand{\Linp}{\Lin^{\perp}}

\title{Stochastic Processes problems}
\author{\url{https://github.com/bdemeshev/stochastic_pro}}
\date{\today}


%\newtheorem{problem}{Задача}
%\numberwithin{problem}{section}

\Newassociation{sol}{solution}{solution_file}
% sol --- имя окружения внутри задач
% solution --- имя окружения внутри solution_file
% solution_file --- имя файла в который будет идти запись решений
% можно изменить далее по ходу
\Opensolutionfile{solution_file}[all_solutions]
% в квадратных скобках фактическое имя файла



% магия для автоматических гиперссылок задача-решение
\newlist{myenum}{enumerate}{3}
% \newcounter{problem}[chapter] % нумерация задач внутри глав
\newcounter{problem}[section]

\newenvironment{problem}%
{%
\refstepcounter{problem}%
%  hyperlink to solution
     \hypertarget{problem:{\thesection.\theproblem}}{} % нумерация внутри глав
     % \hypertarget{problem:{\theproblem}}{}
     \Writetofile{solution_file}{\protect\hypertarget{soln:\thesection.\theproblem}{}}
     %\Writetofile{solution_file}{\protect\hypertarget{soln:\theproblem}{}}
     \begin{myenum}[label=\bfseries\protect\hyperlink{soln:\thesection.\theproblem}{\thesection.\theproblem},ref=\thesection.\theproblem]
     % \begin{myenum}[label=\bfseries\protect\hyperlink{soln:\theproblem}{\theproblem},ref=\theproblem]
     \item%
    }%
    {%
    \end{myenum}}
% для гиперссылок обратно надо переопределять окружение
% это происходит непосредственно перед подключением файла с решениями





\begin{document}

\maketitle % ставим сюда название, автора и время создания

% здесь нужна прикольная картинка

\newpage
\tableofcontents{}

\newpage


\section{First step analysis}

\begin{problem}
Biden and Trump alternately throw a fair dice infinite number of times. 
Biden throws first. 
The person who obtain the first $6$ wins the game. 

\begin{enumerate}
  \item What is the probability that Biden will win?
  \item What is expected number of turns?
  \item What is variance of turns?
  \item What is expected number of turns given that Biden won?
  \item Find the transition matrix of this four state Markov chain. 
\end{enumerate}

\begin{sol}

\end{sol}
\end{problem}



\begin{problem}
  Elon throws an unfair coin until ``head'' appears. 
  The probability of ``head'' is $p \in (0;1)$. 
  Let $N$ be the total number of throws. 
  \begin{enumerate}
    \item Find $\E(N)$, $\E(\exp(tN))$, $\Var(N)$, $\E(N^3)$.
    \item What is the probability than $N$ will be even?
  \end{enumerate}

  \begin{sol}
  
  \end{sol}
\end{problem}
  


\begin{problem}
  Alice and Bob throw a fair coin until the sequence $HTT$ or $THT$ appears.
  Alice wins if $HTT$ appears first, Bob wins if $THT$ appears first. 
  \begin{enumerate}
    \item Find the probability that Alice wins.
    \item Find the expected value and variance of the total number of throws. 
    \item Using any open source software find the probability that Alice wins for all possible combinations 
    of three coins sequences for Alice and Bob. 
    \item Now Alice and Bob play the following game. 
    Alice chooses her three coins winning sequence first. Next Bob, knowing the choice of Alice, chooses his three coins winning sequence. 
    Than they throw a fair coin until either of their sequences appears. 
    What is the best strategy for Alice? For Bob? What is the probability that Alice wins this game?
  \end{enumerate}
  
  \begin{sol}
  
  \end{sol}
\end{problem}

\begin{problem}
You throw a dice unbounded number of times. 
If it shows $1$, $2$ or $3$ then the corresponding amount of dollars 
is added in the pot. 
It it shows $4$ or $5$ the game stops and you get the pot. 
If it shows $6$ the game ends and you get nothing. 
Initially the pot is empty. 

\begin{enumerate}
  \item What is probability that the game will end by $6$?
  \item What is expected duration of the game?
  \item What is your expected payoff?
  \item What is your payoff variance?
  \item Consider variation-A of the game. 
  Rules are the same, but initially the pot contains $100$ dollars. 
  How will the answers to questions (a)-(d) change?
  \item Consider variation-B of the game. 
  Initially the pot is empty. One rule is changed. 
  If the dice shows $5$ the content of the pot is burned and the game continues. 
  How will the answers to questions (a)-(d) change?
\end{enumerate}

  \begin{sol}
  \end{sol}
\end{problem}


\begin{problem}
Boris Johnson throws a fair coin until $1$ appears or until he says ``quit''.
His payoff is the value of the last throw. 
Boris optimizes his expected payoff. 
If many strategies gives the same expected payoff 
he chooses the strategy that minimizes the expected duration of the game. 

\begin{enumerate}
  \item What is the optimal strategy and the corresponding expected payoff?
  \item What is the expected duration?
  \item How the answers to points (a) and (b) will change
  if Boris should pay $0.3$ dollars for each throw?
  \end{enumerate}

  \begin{sol}
  
  \end{sol}
\end{problem}


\begin{problem}
  
  \begin{sol}
  
  \end{sol}
\end{problem}


\begin{problem}
  
  \begin{sol}
  
  \end{sol}
\end{problem}

\begin{problem}
  
  \begin{sol}
  
  \end{sol}
\end{problem}




% для гиперссылок на условия
% http://tex.stackexchange.com/questions/45415
\renewenvironment{solution}[1]{%
         % add some glue
         \vskip .5cm plus 2cm minus 0.1cm%
         {\bfseries \hyperlink{problem:#1}{#1.}}%
}%
{%
}%

\section{Solutions}
\protect \hypertarget {soln:1.1}{}
\begin{solution}{{1.1}}

\end{solution}
\protect \hypertarget {soln:1.2}{}
\begin{solution}{{1.2}}

  
\end{solution}
\protect \hypertarget {soln:1.3}{}
\begin{solution}{{1.3}}

  
\end{solution}
\protect \hypertarget {soln:1.4}{}
\begin{solution}{{1.4}}
  
\end{solution}
\protect \hypertarget {soln:1.5}{}
\begin{solution}{{1.5}}

  
\end{solution}
\protect \hypertarget {soln:1.6}{}
\begin{solution}{{1.6}}

  
\end{solution}
\protect \hypertarget {soln:1.7}{}
\begin{solution}{{1.7}}

  
\end{solution}
\protect \hypertarget {soln:1.8}{}
\begin{solution}{{1.8}}

  
\end{solution}



\printindex


\section{Sources of wisdom}

\nocite{buzun2015stochastic}

\printbibliography[heading=none]


\end{document}
